\documentclass[a4paper]{article}
\usepackage{lingmacros}
\usepackage[utf8]{inputenc}
\usepackage{bbm}
\usepackage{amsmath}

\usepackage{tree-dvips}
\usepackage{cases}



\title{1セクション演習解答}
\author{B3 浅野光平}
\date{2019/08/17}
\setcounter{section}{1}
% \setcounter{subsection}{1}
% \setcounter{subsubsection}{a}

\begin{document}

\maketitle
\subsection{位数2の剰余群が体になることを示す}
\begin{align}
  1+0&=0+1=1\qquad&\mbox{(加法単位元)}\notag \\
  1+1&=0+0=0\qquad&\mbox{(加法逆元)}\notag \\
  1\times0&=0\times1=0\qquad&\mbox{(乗法単位元)}\notag \\
  1\times1&=1\qquad&\mbox{(乗法逆元)}\notag
\end{align}


\subsection{位数2\mathbb{F}_2}
\renewcommand{\thesubsubsection}{\qquad \alph{subsubsection}}
\subsubsection{多項式$g(x,y)=x^2y+y^2x\in\mathbb{F}_2[x,y]$}

\begin{table}[htbp]
  \centering
    \begin{tabular}{l|c|r}
       $x$ & $y$ & $x^2y+y^2x$  \\ \hline\hline
       $0$ & $0$ & $0^2\times0+0^2\times0=0$ \\
       $0$ & $1$ & $0^2\times1+1^2\times0=0$ \\
       $1$ & $0$ & $1^2\times0+0^2\times1=0$ \\
       $1$ & $0$ & $1^2\times1+1^2\times1=0$ \\ \hline
    \end{tabular}
\end{table}
\forall$(x,y)\in\mathbb{F}^2, g(x,y) = 0$.有限体なので0多項式と任意の点で消えることは同値ではない

\end{document}
