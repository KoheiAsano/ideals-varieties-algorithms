\documentclass[a4paper]{article}
\usepackage{lingmacros}
\usepackage[utf8]{inputenc}
\usepackage{amsmath}
\usepackage{tree-dvips}
\usepackage{cases}
\usepackage{bbm}

\title{1セクション演習解答}
\author{B3 浅野光平}
\date{2019/08/17}
\setcounter{section}{1}
\setcounter{subsection}{1}
% \setcounter{subsubsection}{1}

\begin{document}

\maketitle
\subsection{位数2の剰余群が体になることを示す}
\begin{align}
  1+0&=0+1=1\qquad&\mbox{(加法単位元)}\notag \\
  1+1&=0+0=0\qquad&\mbox{(加法逆元)}\notag \\
  1*0&=0*1=0\qquad&\mbox{(乗法単位元)}\notag \\
  1*1&=1\qquad&\mbox{(乗法逆元)}\notag
\end{align}


\subsection{位数2の剰余群が体になることを示す}



% index付きの数式
\begin{align}
  f(x) &= x^2\\
  f'(x) &=2x\\
  F(x) &= \int f(x)dx\\
  F(x) &= \frac{1}{3x^3}
\end{align}

\[
  \vec{a} = (a_1, a_2, \cdots, a_n)
\]
% set operations
\begin{align}
f,g\in k[x_1,x_2]
h_i\in k[x_1,x_2,...,x_n] & for 1<i<s
\end{align}

\begin{numcases}
  { }
  2x+4y=10 & \\
  x+3y=6 &
\end{numcases}
\[
  \bm{A} = \left(
    \begin{array}{c}
      a_1 \\
      a_2 \\
      \vdots \\
      a_n
    \end{array}
  \right)
\]

% 最大
\[
  m = \max \{ f(x), x = 1, 2, \cdots, X \}
\]

% 最小
\[
  m = \min \{ f(x), x = 1, 2, \cdots, X \}
\]

\end{document}
